%%%%%%%%%%%%%%%%%%%%%%%%%%%%%%%%%%%%%%%%%%%
\newpage
\chapter{Discrete Dislocation Dynamics  in MoDELib}



\section{Topology of dislocation networks}

\subsection{Topological operations}



The objective of this chapter is to give an overview of the DDD method, and to describe its implementation in the MoDELib library. 

DDD is a simulation method to study the plastic deformation of crystalline materials due to the motion of dislocations. 
It is a hybrid discrete-continuum simulation method, which  in the multiscale modeling framework bridges   atomistic  methods (e.g. Molecular Dynamics) and continuum methods (e.g. Finite Element Crystal Plasticity). 
Its hybrid character is due to the fact that crystal dislocations are represented discretely (individually), although their interactions are computed semi-analytically using continuum elasticity and without atomistic degrees of freedom.
 
DDD simulation typically take place in a spatial domain representing either a finite or a periodic material volume. Fig.~\ref{DDDnutshellA} shows a snapshot of a typical MoDELib DDD simulation, where individual dislocation lines are colored according to their Burgers vector, and the intensity of the shaded background represents the local slip due to their motion.  A minimal DDD algorithm is reported in Fig.~\ref{DDDnutshellB}. 


\SetKwFor{MyFor}{main loop}{}{end}
\SetKwFor{MyInit}{initialization:}{}{end}
\SetKwFor{MyDisc}{initialize:}{}{end}

\SetKwFor{ForRegion}{\textcolor{blue}{for each region}}{}{end}
\SetKwFor{ForRegionBnd}{\textcolor{blue}{for each region boundary}}{}{end}

 \begin{figure}[h]
 \centering
  \begin{subfigure}{.48\textwidth}
 \includegraphics[width=\textwidth]{image_650}
 \caption{sample configuration}
  \label{DDDnutshellA}
\end{subfigure}
  \begin{subfigure}{.48\textwidth}
%\begin{minipage}[t]{0.45\textwidth}
\begin{algorithm}[H]
%\SetAlgoLined
\MyInit{}
 {
simulation domain\;
material properties and slip systems\;
initial dislocation configuration\;
 }
\MyFor{}{
compute mutual dislocation stress\;
add other stress sources\;
compute line velocities from local stress\;
compute  nodal velocities\;
update nodal positions\;
perform discrete events:\\
- junctions\\
- cross-slip\\
- network remeshing\\
- nucleation\\
 }
\end{algorithm}
 \caption{minimal DDD algorithm}
     \label{DDDnutshellB}
\end{subfigure}
  \caption{typical  DDD simulation in MoDELib. }
    \label{DDDnutshell}
 \end{figure}
 
 During the initialization phase a mesh of the simulation domain is prepared, material properties are selected, and an initial dislocation microstructure is generated. In. MoDELib, these processes are automated as explained in section \ref{runningMoDELib}. During the main simulation loop the mutual interaction forces between dislocations are computed using semi-analytical expressions derived from the elastic theory of dislocations described in section \ref{elasticTheoryDislocations}. This is typically the most time-consuming part of a single simulation step. External loads and possibly other stress sources are then added to determine the total force on each line element of a dislocation. Overdamped dynamics is then invoked to compute the local dislocation velocity given the local force, using special functions called \textit{mobility laws}. A Finite Element (FE) scheme is then employed to find the velocity of the discretization points (nodes) which determine the dislocation configuration. Once ta new configuration is obtained through a simple time-marching scheme, a series of processes takes place to update the dislocation topology due to discrete physical events such as collisions, cross slip, etc.

In the remainder of this chapter we develop the necessary theoretical background necessary to understand the  DDD method. We then proceed to the description of the MoDELib implementation. 

\newpage
\section{Uniform load controller}

The UniformLoadController class is used to apply a uniform external stress state to the dislocations in the RVE.  The RVE is assumed to be subjected to two loads, an applied stress $\bm \sigma_0$, and a stress $\bm \sigma_s$ exerted by a ``multiaxial spring". Assuming Voigt notation throughout, the spring has \emph{diagonal} stiffness $\bm D$, and it is under an applied strain $\bm \varepsilon_0$, so that $\bm \sigma_s=\bm D\left(\bm \varepsilon_0-\bm\varepsilon\right)$, where $\bm\varepsilon$ is the strain in the RVE. 
Therefore the stress in the RVE is
\begin{align}
\bm \sigma=\bm \sigma_0+\bm \sigma_s= \bm \sigma_0+\bm D\left(\bm \varepsilon_0-\bm\varepsilon\right)
\label{ULC1}
\end{align} 
On the other hand, the RVE stress $\bm \sigma$ is  related to the RVE strain $\bm \varepsilon$ by the relation
\begin{align}
\bm \sigma=\bm C\left(\bm \varepsilon-\bm\varepsilon_p\right)
\label{ULC2}
\end{align} 
where $\bm C$ is the material stiffness, and $\varepsilon_p$ is the plastic strain in the RVE. From \eqref{ULC1} and \eqref{ULC2}, we see that the RVE strain is
\begin{align}
\bm \varepsilon&=\left(\bm C+\bm D\right)^{-1} \left[\bm \sigma_0+\bm D\bm \varepsilon_0+\bm C\bm\varepsilon_p\right]
\end{align}
while the RVE stress is
\begin{align}
\bm \sigma&=\bm C\left(\bm C+\bm D\right)^{-1} \left[\bm \sigma_0+\bm D\left(\bm \varepsilon_0- \bm\varepsilon_p\right)\right]\nonumber\\
\end{align}

We take the diagonal values of $\bm D$ to be $D_i=\alpha_i C_{ii}$ (no summation over $i$), where $\alpha_i$ is an array called ``\emph{machine stiffness ratio}". To understand the effects of the machine stiffness ratio, consider the two limits of small and large $\{\alpha_i\}$'s.

First, suppose that $\alpha_i=0$ for some $i$. Let us rewrite stress as
\begin{align}
\bm \sigma&=\underbrace{\left(\bm I+\bm D\bm C^{-1}\right)^{-1}}_{\bm A}\bm \sigma_0+\underbrace{\left(\bm I+\bm D\bm C^{-1}\right)^{-1}\bm D}_{\bm B}\left(\bm \varepsilon_0- \bm\varepsilon_p\right)
\end{align}
 Note that the $i$-th row of the matrix $\bm I+\bm D\bm C^{-1}$ has a 1 in position $i$ and zeros everywhere else. This implies that also the $i$-th row of the matrix $\bm A$ has a 1 in position $i$ and zeros everywhere else, and in turn that  the $i$-th row of $\bm B$ is zero. This means $\sigma_{i}= \sigma_{0i}$. In other words, setting the $i$-th component of the machine stiffness ratio to zero implies that the corresponding stress component $\sigma_{i}$  matches the applied stress component $\sigma_{0i}$.

Next, consider $\alpha_i\rightarrow\infty$ for some $i$. Let us rewrite strain as
\begin{align}
\bm \varepsilon=\underbrace{\left(\bm I+\bm D^{-1}\bm C\right)^{-1}\bm D^{-1}}_{\bm E} \left[\bm \sigma_0+\bm C\bm\varepsilon_p\right]
+\underbrace{\left(\bm I+\bm D^{-1}\bm C\right)^{-1}}_{\bm F} \bm \varepsilon_0
\end{align}
Here we simply note that the matrices $\bm E$ and $\bm F$ behave similar to the matrices $\bm B$ and $\bm A$ above, respectively. Hence the  $\varepsilon_{i}= \varepsilon_{0i}$. In other words, setting the $i$-th component of the machine stiffness ratio to $\infty$ implies that the corresponding strain component $\varepsilon_{i}$  matches the applied strain component $\varepsilon_{0i}$.
%where $\bm D=\text{diag}{\{D_i\}}$ is a \emph{diagonal} matrix of ``machine stiffness ratios". With this choice we obtain
%\begin{align}
%\bm \varepsilon=\bm C^{-1}\left(\bm I+\bm D\right)^{-1} \left(\bm \sigma_0+\bm C\bm D\bm \varepsilon_0+\bm C\bm\varepsilon_p\right)
%\end{align}
%\begin{align}
%\bm \sigma=\left(\bm I+\bm D\right)^{-1} \bm \sigma_0 + \left(\bm I+\bm D\right)^{-1} \bm D\bm C \left(\bm \varepsilon_0- \bm\varepsilon_p\right)
%\end{align}

%\begin{align}
%\bm \varepsilon=\bm C^{-1}\left(\bm I+\bm D\right)^{-1}\left[\bm \sigma_0+\bm D\bm \varepsilon_0+\bm C\bm\varepsilon_p\right]
%\end{align}

%Method 2
%Now, we assume that $\bm D=\bm C\bm D$, where $\bm D=\text{diag}{\{D_i\}}$ is a \emph{diagonal} matrix of ``machine stiffness ratios". With this choice we obtain
%\begin{align}
%\bm \sigma=\bm C\left(\bm I+\bm D\right)^{-1}\bm C^{-1} \left[\bm \sigma_0+\bm C\bm D\left(\bm \varepsilon_0- \bm\varepsilon_p\right)\right]
%\end{align}



